The project specifications require two distinct performance 
measures to judge the quality of the segmentation produced 
by the program:

\begin{enumerate}
    \item Mean average precision (mAP) for ball localization
    \item Mean intersection over union (mIoU) for the class segmentation
\end{enumerate}

The second one is relatively straightforward, it involves 
averaging the intersection over union of each class, where 
the IoU is defined as the ratio between the intersection 
of the prediction and the ground truth and their union.
This is easy to do for both types of predictions produced in 
our program, the ones represented as binary masks and the 
ones represented as \verb|Rect|s (corresponding to a predicted 
ball location).

The first one is more complex, as it requires to build an 
approximated precision-recall curve of the results produced 
by the classifier. This is achieved by storing a rolling 
estimate for both value calculated on an increasing subset 
of the predictions, where a true positive is defined as 
a prediction with an IoU of at least 0.5.
This prediction is stored as a \verb|std::map| of discrete 
points and the AP is estimated using the PASCAL VOC method 
outlined in the article linked in the project specifications.