\section{Conclusions and Future Work}


The developed system has achieved satisfactory performance in segmentatation of table, balls, background and balls localization, especially considering 
that it is exploiting only traditional computer vision techniques without the use of Machine Learning or Deep Learning.
The ball identification step instead, showed major flaws in multiple cases, bringing down the complexive system performance. 
Despite this, the generated 2D maps proved to represent fairly well the game events on each frame, leading to the creation of a modified version
of the given videos in which useful informations are shown. 
\newline \\
To improve the system, every sub-task solution could be refined.
In particular, one way to better the performance could be the development of a method to generate accurate bounding boxes
of the tracked objects at any time. In fact, our system always returns boxes in their starting shape which can change during the video due to ball
movement combined with perspective effects, leading to possibile false positives (IoU between true and predicted below $50\%$ while the system still 
counts it as a ball).
\newline \\
Other than that, another possibile way in which the project solution could be further improved is the adaptation to live-streaming billiard games. In this case the program should be modified
in order to process the video frames in real-time. Additionaly, in order not to lose any frame, strict time constraints on the images processing shall be added. 
