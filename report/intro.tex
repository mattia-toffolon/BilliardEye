\section{Introduction}

In this first section, the main purpose of the project and how the developed system has been evaluted will be presented.


\subsection{Problem specifics}
The aim of this project is to develop a computer vision system for analyzing video footage of various “Eight Ball”
billiard game events by providing high-level information about the status of the match for each video frame in the form of a 
2D top-view minimap.
\newline
More in detail, the system should be able to:
\begin{itemize}
    \item Recognize and localize all the balls inside the playing field, distinguishing them based on their category (1-the
    white “cue ball”, 2-the black “8-ball”, 3-balls with solid colors, 4-balls with stripes);
    \item Detect all the main lines (boundaries) of the playing field;
    \item Segment the area inside the playing field boundaries detected in point 2 into the following categories: 1-the
    white “cue ball”, 2-the black “8-ball”, 3-balls with solid colors, 4-balls with stripes, 5-playing field;
    \item Represent the current state of the game in a 2D top-view visualization map, to be updated at each new frame
    with the current ball positions and the trajectory of each ball that is moving;
    \item Create a modified version of each video in which the generated 2D maps are 
    overlayed on a corner of the respective frames.
\end{itemize}

\noindent
The available dataset consists of 10 folders each one corresponding to a different game event and containing:
\begin{itemize}
    \item a \verb|.mp4| file corresponding to the event video;
    \item a \verb|.png| file corresponding to the first frame of the video;
    \item a \verb|.png| file corresponding to the last frame of the video.
\end{itemize}

\subsection{Performance evaluation}
To evaluate the system performance, the metrics that should be adopted are:
\begin{itemize}
    \item mean Average Precision (mAP);
    \item mean Intersection over Union (mIoU).
\end{itemize}
\begin{flushleft}
Further insights on how these values were computed and used can be found in Section \ref{sec:performance}.
\end{flushleft}


\subsection{Sub-task identification}
To develop the system, the problem was divided into the following sub-tasks:
\begin{itemize}
    \item Table segmentation;
    \item Table sides recognition;
    \item Balls localization and segmentatation;
    \item Balls identification;
    \item Balls tracking;
    \item Map rendering;
    \item System performance evaluation.
\end{itemize}



\subsection{Used material}
If not specified otherwise in the relative section, the developed algorithms always operate exclusively on the given \verb|.mp4| files and
eventual data generated by the program itself. The only additional external material used was the empty minimap of the billiard table which is necessary for
the requested video rendering.