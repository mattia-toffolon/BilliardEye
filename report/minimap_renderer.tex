\section{Minimap Rendering}

The renderer for the minimap was developed in order for it to use the videoPlayer and 
tracker directly. In order to represent the balls correctly the perspective transformation
matrix, that maps the balls to a predermined rectangle with top left side $(0,0)$. We decided
the dimension of such rectangle to be fixed to $500 \times 1000$ pixels, as the usual ratio
for a billiard table is $1:2$. For the dimension of the balls we used $\frac{1}{70}$ the minimap
columns. This was decided empirically on the videos provided, however such radius does not work
with the same precision on all videos, as there are multiple standards for the billiard table 
dimensions.\par 
For the rendering one buffer is kept for the trajectories, at each frame the line between the
previous and current position is drawn as a black line. this buffer is then copied and the
balls are added. To this the png representing the borders is superimposed (the resizing is hard-coded
as the dimensions are always the same), this needs the executable to be always in the same relative
path as the image is contained in the data folder.\par 
Another role of the render is to determine whether a ball went into a hole or not. We decided to
give this role to the this class, even though architecturally it wouldn't be correct, as it is the 
only one that has both knowledge of the trackers and of the actual positions of the balls on the table.
The method used is to determine the closeness of the center of the bounding box of the ball to one 
of the centers. This verification is done only to balls that have moved in the current frame. The
threshold used is 20 pixels, determined empirically. This method is however not perfect, as the 
tracker may behave unexpectedly when a ball enters a hole, as the object it is tracking is disappearing.
This also is problematic when the hole is occluded by the table border itself, as the tracking may fail
to be precise. Another problem is when a ball is particularly close to a hole but not inside, as it
might be falsely put into a hole. If the renderer detects a ball entering a hole it communicates
this to the tracker which drops the ball.
We also decided to generate a struct for the table.png in order for the executable to not be dependent on the position in the filesystem.
