The implemented algorithm that solves the ball localization and segmentation task 
goes through the following three major steps:
\begin{itemize}
    \item Region proposals generation
    \item False positive filtering
    \item Bounding boxes refinement
\end{itemize}


\subsection*{Region proposals generation}
This first step of the algorithm aims to generate regions of interest that will very likely 
contain a ball. The idea behind this step is to find circular regions in the image and create fo each one of them
a bounding box to contain it. To do this, the Hough Transform algorithm applied to circles was adopted
(HoughCircles in OpenCV).

Although, the algorithm wasn't applied directly on the grayscale version of the image since
this would have lead to a great number of false positives and false negatives. The following
pre-processing was applied to the image.
Firstly, the mask obtained through the table segmentation was used isolate the table in the image
in order to constrain HoughCircles to detect circles only inside the table. \\
(IMAGE)

Secondly, to ease to detection of circles relative to balls in the table, the following local operation applied to the image pixels
proved to be very effective. Since the table area is mostly empty, the pixel intensity of the table
was evaluated as the mean intensity of non-zero pixels in the masked image. After this, each non-zero pixel
was set to the absolute value of the difference between the pixel intensity and the evaluated table one.
In this way the all table pixels were set at a value close to zero and instead the balls at a higher intensity
across at least one of the three channels due to differences from the table.
This results in an alternative version of the image in which the table is darker and the transition
between table and balls is much more marked. \\
(IMAGE)

After this, the image was converted to grayscale and HoughCircles was used on it with a general range
of parameters (Method: HOUGH\_GRADIENT, Radius range: 5-15, Distance between circles: img.rows/32, Accumulator threshold: 12).
Working only with the BGR version of the image proved to be unsufficient to find all balls because of the 
overreliance on the color (hue) of the table that made impossibile to detect some balls. To solve this issue
the previous steps are performed a second time on the HSV version of the image in order to focus more on 
the brightness. 
Many balls were detected in both iterations so a smart way to merge the detected circles was necessary.
The final vector of circles is generated by adding all the ones found from the BGR version of the image that proved 
to generate more accurate ball contours and every circles obtained from the HSV version that do not overlap with circles
in the first set.  
Circles were then converted into bounding boxes by creating a Rect object for each circle with abscissa and ordinate equal to the circle ones minus the circle radius,
width and height equal to double the circle radius (diameter).
Through this procedure, for each image a set of bounding boxes is generated, with the guarantee that it contains exactly one bounding box per ball. \\
(IMAGE)


\subsection*{False positive filtering}
