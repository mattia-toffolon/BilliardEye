The implemented algorithm that solves the ball localization and segmentation task 
goes through the following three major steps:
\begin{itemize}
    \item Region proposals generation
    \item False positive filtering
    \item Bounding boxes refinement
\end{itemize}


\subsection*{Region proposals generation}
This first step of the algorithm aims to generate regions of interest that will very likely 
contain a ball. The idea behind this step is to find circular regions in the image and create fo each one of them
a bounding box to contain it. To do this, the Hough Transform algorithm applied to circles was adopted
(\textit{HoughCircles()} in OpenCV).

Although, the algorithm wasn't applied directly on the grayscale version of the image since
this would have lead to a great number of false positives and false negatives. The following
pre-processing was applied to the image.
Firstly, the mask obtained through the table segmentation was used isolate the table in the image
in order to constrain HoughCircles to detect circles only inside the table. \\
(IMAGE)

Secondly, to ease to detection of circles relative to balls in the table, the following local operation applied to the image pixels
proved to be very effective. Since the table area is mostly empty, the pixel intensity of the table
was evaluated as the mean intensity of non-zero pixels in the masked image. After this, each non-zero pixel
was set to the absolute value of the difference between the pixel intensity and the evaluated table one.
In this way the all table pixels were set at a value close to zero and instead the balls at a higher intensity
across at least one of the three channels due to differences from the table.
This results in an alternative version of the image in which the table is darker and the transition
between table and balls is much more marked. \\
(IMAGE)

After this, the image was converted to grayscale and HoughCircles was used on it with a general range
of parameters (Method: HOUGH\_GRADIENT, Radius range: 5-15, Distance between circles: img.rows/32, Accumulator threshold: 12).
Working only with the BGR version of the image proved to be unsufficient to find all balls because of the 
overreliance on the color (hue) of the table that made impossibile to detect some balls. To solve this issue
the previous steps are performed a second time on the HSV version of the image in order to focus more on 
the brightness. 
Many balls were detected in both iterations so a smart way to merge the detected circles was necessary.
The final vector of circles is generated by adding all the ones found from the BGR version of the image, that proved 
to generate more accurate ball contours, and every circles obtained from the HSV version that do not overlap with circles
in the first set.  
Circles were then converted into bounding boxes by creating a Rect object for each circle with abscissa and ordinate equal to the circle ones minus the circle radius,
width and height equal to double the circle radius (diameter).
Through this procedure, for each image a set of bounding boxes is generated, with the guarantee that it contains exactly one bounding box per ball. \\
(IMAGE)


\subsection*{False positive filtering}
As it can be noted from the previous image, to be able to propose regions that cover the whole set of balls in the table, many 
false positive are generated. This happens since there are many circluar regions in the table that do not belong to any ball.
The latters can come from two classes: circles relative to holes and ones belonging to the player hand and arm if present 
inside the table. Using these informations, the following algorithm to filter the false positives was created and implemented.

The filter is composed by two sub-filters placed in cascade to detect the false positives.
The first one operates in the following manner. Since false positive relative to holes are obviously on the table
border, and false positives generated by the player are never isolated (distant from other bounding boxes) since
many circles around the arm and hand are always detected, the filter defines an area inside the projection of the table that excludes
the table borders ("safe area") and filters all isolated (non overlapping) bounding boxes which projected center is inside of such area.
To assure that a bounding box is truly isolated, just for this step all boxes width were expanded by a factor $1.5$ so that closed but naturally
non-overlapping boxes will now overlap. \\
(IMAGE)

The second and final filter exploits the following property. By dividing the image into connected components, one can notice how found regions belonging to holes, table borders and player arm always
touch the borders of the table segmentation. This information was exploited to implement the following procedure.
The image is converted in HSV format and the Canny algorithm is applied to the image in order to detect edges.
Then, the projection of the image is obtained using the trasformation matrix returned by the \textit{getTransformation()} function.
This image is then given to the \textit{connectedComponentsWithStats()} method (using Spaghetti labeling algorithm) to obtain
the labeled regions along with useful statistic such the extremal up, bottom, left and right positions of the region.
In this way, a new image is obtained from the transformed canny one by setting at zero all pixels belonging to regions that touch the image
borders or are labeled as background (label 0).
At this point, the percentage of non-zero pixels in each patch of this last image defined by the bounding boxes is computed.
following our assumptions all boxes relative to false positives will now contained an amount of non-zero pixels close to zero.
A threshold at 24\%, proved to be effective.

Although, this method isn't perfect since it can produce false negatives. In fact, in case of balls close to the borders, their area counted as connected to a component 
of the table border, therefore the respective pixels in the image were set to zero. To fix this kind of errors, a second round of this
procedure is performed but on a new version of the image obtained by running two iterations of the Opening morphological operator with an elliptic 
structuring element of size $(3,2)$. The idea behind this additional step lies in the assumption that table border and balls regions if connected
are weakly connected components. To ensure that no false positive is filtered as true in this last additional step, the percentages
computed during the first iteration are saved into a vector and used during the second in the following manner. If the relative region is no longer touching the image border
and so in the second run the percentage has greatly increased (at least +30\%), then the bounding box is detected as true positive and added
the set of true positives previously computed. \\
(IMAGE)

Using this filter, only the regions (bounding boxes) relative to the balls were kept, ensuring a number of false positives and false
negatives equal to zero across the whole provided dataset.


\subsection*{Bounding boxes refinement}
As suggested by the previous section titles, the found bounding boxes can only be considered as region proposals, since the circles
found through \textit{HoughCircles} in the first step of the detection are not accurate. This happens because the algorithm was used with a general range of parameters that is supposed to work 
on every possibile image. To obtain the final set of bounding boxes, such regions must be refined.
To do so, we relied once again on \textit{HoughCircles}. To ensure greater precision on the ball circle detection the following operations were carried.
Since in some cases the obtained bounding boxes appeared to be sensibly bigger or smaller than the actual ball shape, to have some reference box dimensions the average
bounding box width was computed. For each box center, a rectangular mask with dimensions equal to the mean box width multiplied by a scaling coefficient (1.1) is applied to 
the image in such point to focus on the surrounding area around of the ball. The scaling is necessary to ensure that the whole ball appears in the mask image, since the previously computed bounding
boxes are often really tight.

After doing so, \textit{GrabCut} algorithm (\textit{grubCut()} in OpenCV) is used to perform foreground extraction in order to obtain an approximate segmentation of the ball in the masked image. 
Then, \textit{HoughCircles} is used on the obtained image with the same set of parameters used before with the exception of the accumulator threshold now set at $10$, and
the radius range that now goes from a third to $70\%$ of the mean bounding box width. The choice on this parameters was made to ensure the detection of circles with true ball dimensions
in the specific image, which were estimated through the mean width computation.

This estimation is valid since at most a few boxes have dimensions sensibly different from the true ones. This information is exploited also to ensure that this refinement
does not worsen the quality of already precise bounding boxes. More precisely, only the bounding boxes for which the detected circle has absolute value of difference between radius and previous box
width halfs greater than $4$ are updated. The update is performed by substituting the previous box with the found circle coverted to bounding box in the manner explained before.
(IMAGE)

It must be noted that sometimes the \textit{GrabCut} algorithm fails to return an accurate ball segmentation since it sometimes consider the shadow of the ball as part of the latter.
In such cases, \textit{HoughCircles} won't find any circles and therefore the relative bounding box won't be updated.
For the same reason, this algorithm is adopted only in this last step and it is not used as backbone of the whole ball localization and segmentation procedure.