\section{Project Structure and Execution Instructions}


The project has been developed using the CMake build system and it is organized with the
following structure:
\begin{itemize}
    \item \verb|include| directory, contains the header source code files;
    \item \verb|src| directory, contains the implementation source code files;
    \item \verb|data| directory, contains the background image used to generate the 2D map;
    \item \verb|samples.zip| file, contains the system input, outputs, predictions and ground truths;
    \item \verb|output_videos.zip| file, contains the produced video renderings;
    \item \verb|CMakeLists.txt|, contains CMake instructions to organize the build of the project. 
\end{itemize}

\noindent
The \verb|sample| directory is structured into \verb|samplei| subfolders, where \verb|i| refers to the sample index (starting from zero).
As mentioned before, each subfolder contains inputs, outputs, prediction and ground truths relative to a specific sample.
\\
\noindent
Furthermore, folders \verb|include| and \verb|src| are further divided into:
\begin{itemize}
    \item \verb|segment| directory, contains the files for table and balls segmentatation;
    \item \verb|recognition| directory, contains the files for balls identification and side recognition;
    \item \verb|tracking| directory, contains the files for balls tracking;
    \item \verb|rendering| directory, contains the files for the map rendering;
    \item \verb|utils| directory, contains additional files that implement various utilities.
\end{itemize}

\noindent
Additionaly, \verb|src| folder contains:
\begin{itemize}
    \item \verb|output_main.cpp| file, implements the full video rendering and masks generation;
    \item \verb|performance_main.cpp|, implements the computation of all performance metrics across all videos.
\end{itemize}

\noindent
After successfully build the project using CMake, the two \textit{mains} can be executed under the following conditions.
\verb|output_main.cpp| requires the following three parameters as argument:
\begin{enumerate}
    \item path to the \verb|.mp4| file corresponding to the video;
    \item path to the \verb|.png| file corresponding to the first video frame;
    \item path to the folder in which the program output shall be saved.
\end{enumerate}

\noindent
\verb|performance_main.cpp| instead, requires the following parameters:
\begin{enumerate}
    \item path to the \verb|samples| folder;
    \item total number of samples in the folder.
\end{enumerate}
\verb|sample| is provided in \verb|.zip| format for convenience. Therefore, in order to run the respective executable
successfully, the \verb|.zip| file must be extracted.
