the points are ordered from the upper-left-most point clockwise, but this doesn't give
any information on how the sides are ordered. It is unfortunately impossible (or out of my
mathematical capabilities) to find any meaningful geometrical relation between the sides
given the perspective deformations in some cases. The only feature we can go off is the 
holes in the middle of the longer sides. The first operation was to obtain the rotated
rectangles containing the sides of the pool table, with a small width as to avoid 
keeping in the images unneeded information.\par 
Multiple approaches were tested to distinguish the two types of sides. The first was 
to use features (such as harris corners or SIFT), as intuitively they should appear 
on the holes and not on uniform sides as they should be classified as edges. This didn't
work, as the features and corners detected most of the time were spread throughout the 
considered sides most of the times.\par
The next approach was to try template matching between different segments of the image,
as the center contains a hole which should make the match significantly worse than
between the left and right segment. This does not work for two reasons:
\begin{itemize}
    \item because of the perspective the hole can appear in a part of the segment of the image that is not the center one
    \item because of the illumination and small imprecisions on the table segmentation the matching would fail even on the segments which should result similar
\end{itemize}
\par
Ultimately the way to recognize the sides was the following:
\begin{itemize}
    \item apply the Canny tranform on the initial image
    \item extract the rectangles from the transformed image
    \item rotate the rectangles so that they are horizontal
    \item apply sobel on the x axis
    \item sum over the contribution on the middle two quarters of the image (normalized over the number of pixels contained)
    \item determine the longer sides by the highest contribution
\end{itemize}
This works because of the nature of the edges on the sides: on shorter sides there
are only parallel lines, instead on longer there are strong lines orthogonal given by the holes.
Only the middle two quarters are considers as to avoid counting the corner holes as edges.
